\section{预备知识}

\subsection{给出ISA+的指令编码设计}

\begin{table}[H]
    \centering
    \begin{tabular}{|c|c|c|c|c|c|c|c|c|c|c|}
        \hline 
        字节 & 0 & 1 & 2 & 3 & 4 & 5 & 6 & 7 & 8 & 9 \\ 
        \hline 
        halt & 0|0 & \multicolumn{9}{c|}{} \\ 
        \hline 
        nop & 1|0 & \multicolumn{9}{c|}{} \\ 
        \hline 
        rrmovq rA,rB & 2|0 & rA|rB & \multicolumn{8}{c|}{} \\ 
        \hline 
        irmovq V,rB & 3|0 & F|rB & \multicolumn{8}{c|}{V} \\ 
        \hline 
        rmmovq rA,D(rB) & 4|0 & rA,rB & \multicolumn{8}{c|}{D} \\ 
        \hline 
        mrmovq D(rB),rA & 5|0 & rA,rB & \multicolumn{8}{c|}{D} \\ 
        \hline 
        OPq rA,rB & 6|fn & rA,rB & \multicolumn{8}{c|}{} \\ 
        \hline 
        jXX Dest & 7|fn & \multicolumn{8}{c|}{Dest} &  \\ 
        \hline 
        cmovXX rA,rB & 2|fn & rA,rB & \multicolumn{8}{c|}{} \\ 
        \hline 
        call Dest & 8|0 & \multicolumn{8}{c|}{Dest} &  \\ 
        \hline 
        ret & 9|0 & \multicolumn{9}{c|}{}  \\ 
        \hline 
        pushq rA & A|0 & rA|F & \multicolumn{8}{c|}{} \\ 
        \hline 
        popq rA & B|0 & rA|F & \multicolumn{8}{c|}{} \\ 
        \hline 
        iaddq V,rB & C|0 & F|rB & \multicolumn{8}{c|}{V} \\ 
        \hline 
    \end{tabular} 
    \caption{ISA+指令集}
\end{table}

\begin{table}[H]
    \centering
   \begin{tabular}{|c|c|c|c|c|c|c|c|c|c|}
       \hline 
       \multicolumn{2}{|c|}{整数操作指令} & \multicolumn{4}{|c|}{分支命令} & \multicolumn{4}{|c|}{传送指令} \\ 
       \hline 
       addq & 6|0 & jmp & 7|0 & jne & 7|4 & rrmovq & 2|0 & cmovne & 2|4 \\ 
       \hline 
       subq & 6|1 & jle & 7|1 & jge & 7|5 & cmovle & 2|1 & cmovge & 2|5 \\ 
       \hline 
       andq & 6|2 & jl & 7|2 & jg & 7|6 & cmovl & 2|2 & cmovq & 2|6 \\ 
       \hline 
       xorq & 6|3 & je & 7|3 &  &  & cmove & 2|3 &  &  \\ 
       \hline 
   \end{tabular} 
   \caption{ISA+指令集功能码}
\end{table}

\begin{table}[H]
    \centering
    \begin{tabular}{|c|c|c|c|c|c|c|c|c|}
        \hline 
        数字 & 0 & 1 & 2 & 3 & 4 & 5 & 6 & 7  \\ 
        \hline 
        寄存器名字 & \%rax & \%rcx & \%rdx & \%rbx & \%rsp & \%rbp & \%rsi & \%rdi  \\ 
        \hline 
        数字 & 8 & 9 & A & B & C & D & E & F \\
        \hline
        寄存器名字 & \%r8 & \%r9 & \%r10 & \%r11 & \%r12 & \%r13 & \%r14 & 无寄存器 \\
        \hline
    \end{tabular} 
    \caption{ISA+寄存器标识符}
\end{table}

\subsection{写出指令iaddq在SEQ中的计算过程}

\begin{enumerate}
    \item 取指令(Fetch)
    \subitem icode:ifun = M1[PC]
    \subitem rA:rB=M1[PC+1]
    \subitem valC=M8[PC+2]
    \subitem valP=PC+10
    \item 译码(Decode)
    \subitem valB=R[rB]
    \item 执行(Excute)
    \subitem valE=valB + valC
    \subitem set CC
    \item 写回(Memory)
    \item 更新PC(Write)
    \subitem R[rB]=valE
\end{enumerate}

\subsection{使用iaddq指令重写教材图4-6中的sum函数}

\begin{minted}{gas}
sum:
    xorq   %rax,%rax
    andq   %rsi,%rsi
    jmp    test
loop:
    mrmovq (%rdi),%r10
    addq   %r10,%rax
    iaddq  $8,%rdi
    iaddq  $-1,%rsi
test:
    jne loop
    ret
\end{minted}


