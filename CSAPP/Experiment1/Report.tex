\documentclass[11pt,a4paper]{ctexart}
% 调整页面大小,默认页面与常用规格不符
\usepackage[margin=1in,top=1.5in]{geometry}
\pagestyle{headings}
\usepackage[utf8]{inputenc}

\usepackage{float}

% 插入数学及电学符号
\usepackage{amsmath}
\usepackage{amsfonts}
\usepackage{amssymb}

% 插入附件
%\usepackage{navigator}

% 引用链接可点击
\usepackage{cite}
\usepackage[colorlinks,linkcolor=black,anchorcolor=blue,citecolor=green]{hyperref}

% 插入图片
\usepackage{graphicx}

% 绘制各种图形,包括电路图,函数图
\usepackage{tikz}
\usepackage{pgfplots}

\makeatletter
\newcommand\dlmu[2][4cm]{\hskip1pt\underline{\hb@xt@ #1{\hss#2\hss}}\hskip3pt}
\makeatother

% 标题居左 显示成如下情况 一、标题
\ctexset{
	section={
		name={第,章},
		number=\arabic{section},
		format=\Large\bfseries\raggedright\heiti
	},
}

% 设置页眉
\usepackage{fancyhdr}
\pagestyle{fancy}
\fancyhf{} % 清空当前的页眉页脚
\fancyhead[C]{\heiti 计算机系统实验报告}

% 用于插入C代码
\usepackage{minted}

\bibliographystyle{plain}
\numberwithin{figure}{section}

\begin{document}
\heiti
\begin{titlepage}
	\centering
	\rule[1.10cm]{\linewidth}{0cm}
	
	\includegraphics[width=0.4\linewidth]{../Banner}
	
	\vspace{0.5cm}
	\textbf{\zihao{0} 实验报告}
	
	\vspace{1cm}
	\textbf{{\Huge 实验(一)}}
	
	\begin{LARGE}
		\vspace{2cm}
		题\rule{38pt}{0pt}目 \dlmu[6cm]{Linux下C工具应用} \\ \vspace{4pt}
		专\rule{38pt}{0pt}业 \dlmu[6cm]{计算机系} \\ \vspace{4pt}
		学\rule{38pt}{0pt}号 \dlmu[6cm]{1160300202}\\ \vspace{4pt}
		班\rule{38pt}{0pt}级 \dlmu[6cm]{1603002}\\ \vspace{4pt}
		学\rule{38pt}{0pt}生 \dlmu[6cm]{冯云龙}\\ \vspace{4pt}
		指导教师 \dlmu[6cm]{刘宏伟}\\ \vspace{4pt}
		实验地点 \dlmu[6cm]{G712}\\ \vspace{4pt}
		实验日期 \dlmu[6cm]{2017/10/10}\\ \vspace{4pt}
	\end{LARGE}
	
	\vfill
	{\huge \textbf{计算机科学与技术学院}}% 底部插入当日日期
\end{titlepage}

\tableofcontents
\newpage
\section{实验基本信息}

\subsection{实验目的}
\begin{enumerate}
    \item 理解现代计算机系统虚拟存储的基本知识
    \item 掌握C语言指针相关的基本操作
    \item 深入理解动态存储申请、释放的基本原理和相关系统函数
    \item 用C语言实现动态存储分配器,并进行测试分析
    \item 培养Linux下的软件系统开发与测试能力
\end{enumerate}

\subsection{实验环境与工具}

\subsubsection{硬件环境}
\begin{itemize}
    \item CPU: Intel Core i7-6700HQ @ 8x 3.5GHz
    \item RAM: 3516MiB / 7899MiB
\end{itemize}

\subsubsection{软件环境}
\begin{itemize}
    \item Manjaro Linux

\end{itemize}

\subsubsection{开发工具}
\begin{itemize}
    \item GCC
    \item Clion
\end{itemize} %实验基本信息
\section{实验环境建立}
\subsection{Linux下CodeBlocks反汇编(10分)}
CodeBlocks运行hellolinux.c。反汇编查看printf函数的实现。
要求:C、ASM、内存(显示hello等内容)、堆栈(call printf前)、寄存器同时在一个窗口。

\begin{figure}[H]
	\centering
	\includegraphics[width=0.55\linewidth]{figures/Linux-Clion}
	\caption{Linux下CLion截图}
	\label{fig:linux-clion}
\end{figure}

\subsection{Linux下EDB运行环境建立(10分)}

用EDB调试hellolinux.c的执行文件,截图,要求同 \ref{fig:linux-clion}。

\begin{figure}[H]
	\centering
	\includegraphics[width=0.5\linewidth]{figures/Linux-EDB}
	\caption{Linux下EDB截图}
	\label{linux-edb}
\end{figure}

 %实验环境建立
\section{分配器的设计与实现}
\begin{center}
    总分50分
\end{center}

\subsection{总体设计(10分)}
介绍堆、堆中内存块的组织结构,采用的空闲块、分配块链表/树结构和相应算法等内容。

\begin{figure}[H]
    \begin{minipage}[l]{0.6\linewidth}
        \heiti
        \begin{tikzpicture}
        \draw (0,0) rectangle +(8,8);
        \draw (7,7) -- (7,8);
        \draw (7,0) -- (7,1);
        \draw (0,1) -- (8,1);
        \draw (0,7) -- (8,7);
        \draw (0,3) -- (8,3);
        \draw (4,5) node{有效载荷};
        \draw (4,2) node{填充(可选)};
        \draw (3.5,0.5) node{块大小};
        \draw (3.5,7.5) node{块大小};
        \draw (7.5,0.5) node{a/f};
        \draw (7.5,7.5) node{a/f};
        \end{tikzpicture}
    \end{minipage}
    \begin{minipage}[c]{0.4\linewidth}
        \heiti
        空闲块通过头部的大小字段隐含的连接着,分配器通过遍历堆中所有的块,从而间接的遍历整个空闲块的集合。
        如果找不到合适的内存块,就对堆的大小进行扩展,需要注意的就是对空闲块的合并,使用了边界标记可以是使得这个操作变得很容易。
    \end{minipage}
\end{figure}

\subsection{关键函数设计(40分)}
\subsubsection{int mm\_init(void)函数(5分)}
\textbf{函数功能:}初始化分配器,成功就返回0,否则返回-1。

\textbf{处理流程:}申请一个四字的块,将第一个字节填充置零,第二、三个字分别作为先导块头部与脚部,设为已分配状态,防止之后的内存块错误合并,第四个字节作为新申请的头部,状态为已占用。将heap\_listp指向初始头部与脚部之间载荷(大小为0)位置。

\textbf{要点分析:}
\subsubsection{void mm\_free(void *ptr)函数(5分)}
\textbf{函数功能:}释放参数"ptr"指向的已分配内存块,没有返回值。指针值ptr应该是之前调用mm\_malloc或mm\_realloc返回的值,并且没有释放过。

\textbf{参   数:}需要被释放的指针。

\textbf{处理流程:}获取该块的大小,将头部与脚部的有效位置零表示空闲,合并空闲块。

\textbf{要点分析:}内存块需要重新标记为未使用。

\subsubsection{void *mm\_realloc(void *ptr, size\_t size)(5分)}
\textbf{函数功能:}调用mm\_realloc是为了将ptr所指向内存块(旧块)的大小变为size,并返回新内存块的地址。

\textbf{参   数:}需要重新分批大小的内存块,重新分配的大小。

\textbf{处理流程:}获取需要重新分配的字节大小,与当前块大小比较,如果小于现有块大小,则直接返回,若现有的大小比较小,则查看能否与前后的空闲块进行合并,否则申请新块并复制旧块内容,释放旧块,返回指向新块的指针。

\textbf{要点分析:}需要注意下面几个方面的要求。
\begin{itemize}
    \item 如ptr是空指针NULL,等价于mm\_malloc(size)
    \item 如果参数size为0,等价于mm\_free(ptr)
    \item 如ptr非空, 它应该是之前调用mm\_malloc或mm\_realloc返回的数值,指向一个已分配的内存块。
\end{itemize}
\subsubsection{int mm\_check(void)函数函数(5分)}
\textbf{函数功能:}函数,检查重要的不变量和一致性条件。当且仅当堆是一致的,才能返回非0值。

\textbf{处理流程:}检查空闲列表中的每个块是否都空闲,是否有连续的空闲块没有被合并,是否每个空闲块都在空闲链表中,空闲链表中的指针是否均指向有效的空闲块,分配的块是否有重叠,堆块中的指针是否指向有效的堆地址。若不一致,返回0,若一致,返回1。

\textbf{要点分析:}重点检查如下几个方面。
\begin{enumerate}
    \item 空闲列表中的每个块是否都标识为free(空闲)?
    \item 是否有连续的空闲块没有被合并?
    \item 是否每个空闲块都在空闲链表中?
    \item 空闲链表中的指针是否均指向有效的空闲块?
    \item 分配的块是否有重叠?
    \item 堆块中的指针是否指向有效的堆地址?
\end{enumerate}

\subsubsection{int void *mm\_malloc(size\_t size)函数(10分)}
\textbf{函数功能:}申请有效载荷至少是参数"size"指定大小的内存块,返回该内存块地址首地址(可以使用的区域首地址)。

\textbf{参   数:}申请分配内存块的大小。

\textbf{处理流程:}若传入参数不大于零,返回NULL。否则令新块大小为头部之前填充与对齐后字节大小之和,查找空闲链表中的空闲块,将新块放入,如没有合适空闲块,申请新的内存。

\textbf{要点分析:}申请的整个块应该在对的区间内,并且不能与其他已经分配的块重叠。返回的地址应该是8字节对齐的(地址\%8==0)。
\subsubsection{static void *coalesce(void *bp)函数(10分)}
\textbf{函数功能:}将要回收的空闲块和临近的空闲块(如果有的话)合并成一个大的空闲块。返回合并后的空闲块指针。

\textbf{参   数:}bp是要回收的空闲块指针

\textbf{处理流程:}获取头部和脚部的标识位,以及块的大小,分四种情况。
\begin{enumerate}
    \item 前后都已分配,返回被释放块的指针
    \item 下一块空闲,将块大小扩展为两块之和,释放块的头部和后一块的脚部标识位设置为0
    \item 前一块空闲,将块大小扩展为两块之和,前一块的头部和释放块的脚部标识位设置为0,指针指向前一块
    \item 前后均空闲,将块大小扩展为三块之和,前一块的头部和后一块的脚部标识位设置为0,指针指向前一块
\end{enumerate}

\textbf{要点分析:}每次对块进行扩展,需注意扩展之后对标识位的设置改变,扩展完毕后,需移动指针位置。 %WINDOWS软硬件系统观察与分析
\section{总结}

\subsection{请总结本次实验的收获}
本次实验,对C语言和汇编之间的关系有了更进一步的了解,对于GDB的使用也变得更加熟悉,感觉收获很大。

\subsection{请给出对本次实验内容的建议}
希望能够提前发PPT和实验报告。 %LINUX软硬件系统观察与分析
\section{以16进制查看程序HELLO.C}
\subsection{请查看HELLOWIN.C与HELLOLINUX.C的编码(3分)}

HelloWin.c采用\dlmu[2cm]{GBK}编码,HelloLinux.c 采用\dlmu{UTF-8}编码,你的姓名\dlmu[2cm]{冯云龙}分别编码为:\dlmu[7cm]{B7\ EB\ D4\ C6\ C1\ FA}\\
与\dlmu[7cm]{E5\ 86\ AF\ E4\ BA\ 91\ E9\ BE\ 99\ 0A}。

HelloWin.c在Linux下用gcc缺省模式编译后运行结果为:\dlmu[5cm]{Hello\ 1160300202\ ??????}。

\subsection{请查看HELLOWIN.C与HELLOLINUX.C的回车(3分)} 

Windows下的回车编码为:\dlmu[2cm]{0D\ 0A},Linux下的回车编码为:\dlmu[2cm]{0A 0A}。
交叉打开文件的效果是\dlmu[6cm]{hellolinux.c全部变成一行},
\dlmu[6cm]{hellowin.c 后面多了一个 $\hat{ } M$ 符号}。

\section{程序的生成CPP、GCC、AS、LD}

%\subsection{请提交每步生成的文件(4分)}
%\embeddedfile[Source file]{hellolinux.c}{src/hello.c}
%\embeddedfile[Preprocess file]{hellolinux.i}{src/hello.o}
%\embeddedfile[Assemble file]{hellolinux.o}{src/hello.c}
%\embeddedfile[Object file]{hellolinux.out}{src/hello.out}

\begin{enumerate}
	\item 预处理,生成预编译文件(.i文件):\mintinline{sh}|Gcc –E hello.c –o hello.i|
	\item 编译,生成汇编代码(.s文件):\mintinline{sh}|Gcc –S hello.i –o hello.s|
	\item 汇编,生成目标文件(.o文件):\mintinline{sh}|Gcc –c hello.s –o hello.o|
	\item 链接,生成可执行文件:\mintinline{sh}|Gcc hello.o –o hello.out|
\end{enumerate}

\inputminted{c}{src/hello.c}

%需要提交的文件在PDF附件当中。 %以16进制查看程序HELLO.C %程序的生成
\section{计算机系统的基本信息获取编程}
\subsection{CPUID信息}
CPUID指令是intel IA32架构下获得CPU信息的汇编指令,可以得到CPU类型,型号,制造商信息,商标信息,序列号,缓存等一系列CPU相关的东西。
cpuid使用eax作为输入参数,eax,ebx,ecx,edx作为输出参数。

把eax = 0作为输入参数,cpuid指令执行以后,会返回一个12字符的制造商信息,前四个字符的ASCI码按低位到高位放在ebx,中间四个放在edx,最后四个字符放在ecx。
把eax = 1作为输入参数,cpuid指令执行以后,会返回CPU的版本、架构和签名,签名信息放在eax,特性放在edx和ecx,附加特性放在ebx。

\subsection{获取MAC}
使用\mintinline{c}|ioctl|和\mintinline{c}|socket|函数获取网卡MAC地址。

socket函数用于创建套接字,之后我们使用这个套接字来获取网卡的MAC地址,I/O控制函数ioctl用于对文件进行底层控制,这里的文件包含网卡、终端、磁带机、套接口等软硬件设施,实际的操作来自各个设备自己提供的ioctl接口。

\mintinline{c}|int socket(int domain,int type, int protocol);|
\begin{itemize}
	\item domain参数:表示所使用的协议族(取值AF\_INET,表示采用internet协议族)
	\item type参数:表示套接口的类型(SOCK\_DGRAM,表示采用数据报类型套接口)
	\item protocol参数:表示所使用的协议族中某个特定的协议(自动选择,用0填充)
\end{itemize}

如果函数调用成功,套接口的描述符(非负整数)就作为函数的返回值,假如返回值为-1,就表明有错误发生。

\mintinline{c}|int ioctl(int fd,int request,…)|。
\begin{itemize}
	\item fd参数:文件描述符(取套接口的描述符)
	\item request参数:指定信息获取码(SIONGIFHWADDR表示取硬件地址)
	\item 其后的request参数用于为实现I/O控制所必须传入或传出的参数
\end{itemize}

本实验需要用ifr结构传入网卡设备名,并传出6B的MAC地址。

\subsection{请提交源程序文件(10分)} 
\inputminted{c}{src/sysinfo.c} %计算机系统的基本信息获取编程
\section{计算机数据类型的本质}
\subsection{源程序文件DATATYPE.C(10分)}

%\embeddedfile[Source file]{datatype.c}{src/datatype.c}

\inputminted{c}{src/datatype.c} %计算机数据类型的本质
\section{程序运行分析}
\subsection{SUM的分析(20分)}

\begin{minted}{c}
int sum(int a[], unsigned len) {
    int i, sum = 0;
    for (i = 0; i <= len - 1; i++)
    sum += a[i];
    return sum;
}
\end{minted}

\subsubsection{运行结果}程序运行后会发生数组越界,而后被系统强行终止。

\subsubsection{原因分析}\mintinline{c}|len|是\mintinline{c}|unsigned|类型,输入\mintinline{c}|len=0|,导致的结果其实是\mintinline{c}|i|始终与\mintinline{c}|0xFFFFFFFF|比较大小,\mintinline{c}|i|是\mintinline{c}|int|类型,进行了类型后转换后始终比\mintinline{c}|len|小,之后在循环过程中就会越来越大,最终数组访问到无权限的位置被强行终止。

\subsubsection{改进方法}修改 \mintinline{c}|for(i = 0;i <= len-1;i++)| 为 \mintinline{c}|for(i = len;i > 0;i++)|。

\subsection{FLOAT的分析(20分)} 
\begin{minted}{c}
#include <stdio.h>

int main() {
  float f;
  for (;;) {
  printf("请输入一个浮点数:");
  scanf("%f", &f);
  printf("这个浮点数的值是:%f\n", f);
  if (f == 0)
    break;
  }
  return 0;
}
\end{minted}

\subsubsection{运行结果}
\begin{tabular}{|c|c|c|c|}
	\hline 
	输入 & 输出 & 输入 & 输出 \\ 
	\hline 
	61.419997 & 61.419998 & 10.186810 & 10.186810 \\ 
	\hline 
	61.419998 & 61.419998 & 10.186811 & 10.186811 \\ 
	\hline 
	61.419999 & 61.419998 & 10.186812 & 10.186812 \\ 
	\hline 
	61.420000 & 61.419998 & 10.186813 & 10.186813 \\ 
	\hline 
	61.420001 & 61.420002 & 10.186814 & 10.186814 \\ 
	\hline
	          &           & 10.186815 & 10.186815 \\ 
	\hline 
\end{tabular} 

\subsubsection{原因分析}
由于\mintinline{c}|float|是用有限的内存存储无限的数据,这当然是不可能的,所以\mintinline{c}|float|的值是离散的,不精确的,在输入60.419998等数据时,精度是不足的,C语言对其进行了舍入,而在输入10.186810时,\mintinline{c}|float|精度足够,所以可以取得精确值。

\subsubsection{注意事项}
C语言中\mintinline{c}|float|的精度是有限的,数值越大,越不精确,所以在使用\mintinline{c}|float|类型时需要考虑精确度的问题。


\section{总结}
\subsection{本次实验的收获}
本次实验,我学会了虚拟机的安装与使用,学会使用C语言和操作系统进行交互,而不仅仅停留于表面,在对程序进行分析的时候,更进一步的明白了C语言的相关特性,获益匪浅。

\subsection{对本次实验内容的建议} 
希望能同时给出 \LaTeX 模板,方便使用。


 %程序运行分析 %总结

\nocite{ref1}
\nocite{ref2}
\nocite{ref3}
\nocite{ref4}
\nocite{ref5}
\nocite{ref6}
\nocite{ref7}

%参考文献
\bibliography{Ref/参考文献}

\end{document}