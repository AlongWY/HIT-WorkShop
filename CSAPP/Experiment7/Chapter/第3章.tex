\section{分配器的设计与实现}
\begin{center}
    总分50分
\end{center}

\subsection{总体设计(10分)}
介绍堆、堆中内存块的组织结构,采用的空闲块、分配块链表/树结构和相应算法等内容。

\begin{figure}[H]
    \begin{minipage}[l]{0.6\linewidth}
        \heiti
        \begin{tikzpicture}
        \draw (0,0) rectangle +(8,8);
        \draw (7,7) -- (7,8);
        \draw (7,0) -- (7,1);
        \draw (0,1) -- (8,1);
        \draw (0,7) -- (8,7);
        \draw (0,3) -- (8,3);
        \draw (4,5) node{有效载荷};
        \draw (4,2) node{填充(可选)};
        \draw (3.5,0.5) node{块大小};
        \draw (3.5,7.5) node{块大小};
        \draw (7.5,0.5) node{a/f};
        \draw (7.5,7.5) node{a/f};
        \end{tikzpicture}
    \end{minipage}
    \begin{minipage}[c]{0.4\linewidth}
        \heiti
        空闲块通过头部的大小字段隐含的连接着,分配器通过遍历堆中所有的块,从而间接的遍历整个空闲块的集合。
        如果找不到合适的内存块,就对堆的大小进行扩展,需要注意的就是对空闲块的合并,使用了边界标记可以是使得这个操作变得很容易。
    \end{minipage}
\end{figure}

\subsection{关键函数设计(40分)}
\subsubsection{int mm\_init(void)函数(5分)}
\textbf{函数功能:}初始化分配器,成功就返回0,否则返回-1。

\textbf{处理流程:}申请一个四字的块,将第一个字节填充置零,第二、三个字分别作为先导块头部与脚部,设为已分配状态,防止之后的内存块错误合并,第四个字节作为新申请的头部,状态为已占用。将heap\_listp指向初始头部与脚部之间载荷(大小为0)位置。

\textbf{要点分析:}
\subsubsection{void mm\_free(void *ptr)函数(5分)}
\textbf{函数功能:}释放参数"ptr"指向的已分配内存块,没有返回值。指针值ptr应该是之前调用mm\_malloc或mm\_realloc返回的值,并且没有释放过。

\textbf{参   数:}需要被释放的指针。

\textbf{处理流程:}获取该块的大小,将头部与脚部的有效位置零表示空闲,合并空闲块。

\textbf{要点分析:}内存块需要重新标记为未使用。

\subsubsection{void *mm\_realloc(void *ptr, size\_t size)(5分)}
\textbf{函数功能:}调用mm\_realloc是为了将ptr所指向内存块(旧块)的大小变为size,并返回新内存块的地址。

\textbf{参   数:}需要重新分批大小的内存块,重新分配的大小。

\textbf{处理流程:}获取需要重新分配的字节大小,与当前块大小比较,如果小于现有块大小,则直接返回,若现有的大小比较小,则查看能否与前后的空闲块进行合并,否则申请新块并复制旧块内容,释放旧块,返回指向新块的指针。

\textbf{要点分析:}需要注意下面几个方面的要求。
\begin{itemize}
    \item 如ptr是空指针NULL,等价于mm\_malloc(size)
    \item 如果参数size为0,等价于mm\_free(ptr)
    \item 如ptr非空, 它应该是之前调用mm\_malloc或mm\_realloc返回的数值,指向一个已分配的内存块。
\end{itemize}
\subsubsection{int mm\_check(void)函数函数(5分)}
\textbf{函数功能:}函数,检查重要的不变量和一致性条件。当且仅当堆是一致的,才能返回非0值。

\textbf{处理流程:}检查空闲列表中的每个块是否都空闲,是否有连续的空闲块没有被合并,是否每个空闲块都在空闲链表中,空闲链表中的指针是否均指向有效的空闲块,分配的块是否有重叠,堆块中的指针是否指向有效的堆地址。若不一致,返回0,若一致,返回1。

\textbf{要点分析:}重点检查如下几个方面。
\begin{enumerate}
    \item 空闲列表中的每个块是否都标识为free(空闲)?
    \item 是否有连续的空闲块没有被合并?
    \item 是否每个空闲块都在空闲链表中?
    \item 空闲链表中的指针是否均指向有效的空闲块?
    \item 分配的块是否有重叠?
    \item 堆块中的指针是否指向有效的堆地址?
\end{enumerate}

\subsubsection{int void *mm\_malloc(size\_t size)函数(10分)}
\textbf{函数功能:}申请有效载荷至少是参数"size"指定大小的内存块,返回该内存块地址首地址(可以使用的区域首地址)。

\textbf{参   数:}申请分配内存块的大小。

\textbf{处理流程:}若传入参数不大于零,返回NULL。否则令新块大小为头部之前填充与对齐后字节大小之和,查找空闲链表中的空闲块,将新块放入,如没有合适空闲块,申请新的内存。

\textbf{要点分析:}申请的整个块应该在对的区间内,并且不能与其他已经分配的块重叠。返回的地址应该是8字节对齐的(地址\%8==0)。
\subsubsection{static void *coalesce(void *bp)函数(10分)}
\textbf{函数功能:}将要回收的空闲块和临近的空闲块(如果有的话)合并成一个大的空闲块。返回合并后的空闲块指针。

\textbf{参   数:}bp是要回收的空闲块指针

\textbf{处理流程:}获取头部和脚部的标识位,以及块的大小,分四种情况。
\begin{enumerate}
    \item 前后都已分配,返回被释放块的指针
    \item 下一块空闲,将块大小扩展为两块之和,释放块的头部和后一块的脚部标识位设置为0
    \item 前一块空闲,将块大小扩展为两块之和,前一块的头部和释放块的脚部标识位设置为0,指针指向前一块
    \item 前后均空闲,将块大小扩展为三块之和,前一块的头部和后一块的脚部标识位设置为0,指针指向前一块
\end{enumerate}

\textbf{要点分析:}每次对块进行扩展,需注意扩展之后对标识位的设置改变,扩展完毕后,需移动指针位置。