\documentclass{ML}
\usepackage{amsthm}
\usepackage{fontspec}
\usepackage[ruled,linesnumbered]{algorithm2e}

\setmonofont{Iosevka Nerd Font Mono}

\newtheorem{theorem}{定理}
% \newtheorem{proof}{证明}

% 姓名,学号
\infoauthor{冯云龙}{20s103221}

% 课程类型,实验名称
\infoexp{选修}{决策树ID3算法}

\begin{document}
\maketitle

\tableofcontents
\newpage

\section{实验目的}

\begin{enumerate}
	\item 掌握最小二乘法求解(无惩罚项的损失函数)
	\item 掌握加惩罚项 (2范数) 的损失函数优化
	\item 掌握梯度下降法
	\item 掌握共轭梯度法
	\item 理解过拟合
	\item 掌握克服过拟合的方法(加惩罚项、增加样本)
\end{enumerate}

\section{实验要求及实验环境}

\subsection{实验要求}

\begin{enumerate}
	\item 生成数据、加入噪声
	\item 用高阶多项式拟合曲线
	\item 用解析解求解两种loss的最优解(无正则项和有正则项)
	\item 优化方法求解最优解(梯度下降、共轭梯度)
	\item 用不同数据集,不同超参数,不同的多项式阶数,比较实验效果
\end{enumerate}

\subsection{实验环境}

\begin{itemize}
	\item 操作系统:Manjaro Linux x64
	\item 编程语言:Python
	\item IDE:PyCharm
\end{itemize}

\section{设计思想}


\subsection{算法原理}

\begin{table}[H]
	\centering
	\begin{tabular}{ccc}
		\\ \hline
		符号                          & 释义           & 注释        \\ \hline
		\(X = x^{1},x^{2},…,x^{m}\)   & 样本集         &             \\
		\(x^i = x_1^i,x_2^i,…,x_n^i\) & 样本属性集     & 通常会包含b \\
		\(Y = y^{1},y^{2},…,y^{m}\)   & 标签集         &             \\
		\(W = w^{1},w^{2},…,w^{n}\)   & 依赖关系描述集 &
		\\ \hline
	\end{tabular}
\end{table}

\subsection{算法的实现}


\section{实验结果与分析}


\section{结论}


% \begin{thebibliography}{0}
%     \addcontentsline{toc}{section}{参考文献}
%     \bibitem{共轭梯度法} 维基百科,{\it \href{https://zh.wikipedia.org/wiki/共轭梯度法}{共轭梯度法词条}}
% \end{thebibliography}

\appendix

\section{源代码}

\inputminted[breaklines=true,frame=lines,mathescape=true]{python}{../decision_tree.py}

\end{document}
