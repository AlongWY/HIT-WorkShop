\section{实验目的}
树型结构的建立与遍历。

\section{实验要求及实验环境}
\subsection{实验要求}
\begin{enumerate}
    \item 至少采用两种方法,编写建立二叉树的二叉链表存储结构的程序,并以适当的形式显示并保存二叉树。
    \item 采用二叉树的二叉链表存储结构,编写程序实现二叉树的先序、中序和后序遍历的递归和非递归算法以及层序遍历算法,以适当的形式显示并保存二叉树和相应的遍历序列。
    \item 在二叉树的二叉链表存储结构基础上,编写程序实现二叉树的中序线索链表存储结构建立的算法,以适当的形式显示并保存二叉树的相应的线索链表。
    \item 在二叉树的线索链表存储结构上,编写程序分别实现求二叉树一个结点的先序、中序和后序遍历的后继结点算法。
    \item 以上条要求为基础,编写程序实现对中序线索二叉树进行先序、中序和后序遍历的非递归算法,以适当的形式显示并保存二叉树和相应的遍历序列。
\end{enumerate}

\subsection{实验环境}
\subsubsection{硬件环境}
\begin{itemize}
    \item CPU: Intel Core i7-6700HQ @ 8x 3.5GHz
    \item RAM: 3516MiB / 7899MiB
\end{itemize}

\subsubsection{软件环境}
\begin{itemize}
    \item Manjaro Linux
\end{itemize}

\subsubsection{开发工具}
\begin{itemize}
    \item GCC
    \item Clion
\end{itemize}