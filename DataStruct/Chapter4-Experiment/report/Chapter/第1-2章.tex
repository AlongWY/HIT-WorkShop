\section{实验目的}
树型结构的建立与遍历。

\section{实验要求及实验环境}
\subsection{实验要求}
\begin{enumerate}
    \item 分别实现有向图的邻接矩阵、邻接表和十字链表存储结构的建立算法,分析和比较各建立算法的时间复杂度以及存储结构的空间占用情况;
    \item 实现有向图的邻接矩阵、邻接表和十字链表三种存储结构的相互转换算法;
    \item 在上述三种存储结构上,分别实现有向图的深度优先搜索(递归和非递归)和广度优先搜索算法。并以适当的方式存储和显示相应的搜索结果(深度优先或广度优先生成森林(或生成树)、深度优先或广度优先序列和编号);
    \item 分析搜索算法的时间复杂度;
    \item 以文件形式输入图的顶点和边,并显示相应的结果。要求顶点不少于10个,边不少于13个;
    \item 软件功能结构安排合理,界面友好,便于使用
\end{enumerate}

\subsection{实验环境}
\subsubsection{硬件环境}
\begin{itemize}
    \item CPU: Intel Core i7-6700HQ @ 8x 3.5GHz
    \item RAM: 3516MiB / 7899MiB
\end{itemize}

\subsubsection{软件环境}
\begin{itemize}
    \item Manjaro Linux
\end{itemize}

\subsubsection{开发工具}
\begin{itemize}
    \item GCC
    \item Clion
\end{itemize}