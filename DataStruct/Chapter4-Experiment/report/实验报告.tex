\documentclass[11pt,a4paper]{ctexart}
% 调整页面大小,默认页面与常用规格不符
\usepackage[margin=1in,top=1.5in]{geometry}
\pagestyle{headings}
\usepackage[utf8]{inputenc}

\usepackage{float}

% 插入数学及电学符号
\usepackage{amsmath}
\usepackage{amsfonts}
\usepackage{amssymb}

% 插入附件
\usepackage{navigator}

% 引用链接可点击
\usepackage{cite}
\usepackage[colorlinks,linkcolor=black,anchorcolor=blue,citecolor=green]{hyperref}

% 插入图片
\usepackage{graphicx}

% 绘制各种图形,包括电路图,函数图
\usepackage{tikz}
\usetikzlibrary{snakes,arrows,shapes}
\usepackage{pgfplots}

\makeatletter
\newcommand\dlmu[2][4cm]{\hskip1pt\underline{\hb@xt@ #1{\hss#2\hss}}\hskip3pt}
\makeatother


% 标题居左 显示成如下情况 一、标题
\ctexset{
section={
name={第,章},
number=\arabic{section},
format=\Large\bfseries\raggedright\heiti
},
}

% 设置页眉
\usepackage{fancyhdr}
\pagestyle{fancy}
\fancyhf{} % 清空当前的页眉页脚
\fancyhead[C]{\heiti 数据结构实验报告}

% 插入pdf文档
\usepackage{pdfpages}

% 用于插入代码
\usepackage{listings}

% 用于绘制流程图
\usepackage{flowchart}

% 调整代码的背景和风格
\usepackage{xcolor}
\lstset{
%行号
numbers=left,
%背景框
framexleftmargin=10mm,
frame=none,
%背景色
%backgroundcolor=\color[rgb]{1,1,0.76},
backgroundcolor=\color[RGB]{245,245,244},
%样式
keywordstyle=\bf\color{blue},
identifierstyle=\bf,
numberstyle=\color[RGB]{0,192,192},
commentstyle=\it\color[RGB]{0,96,96},
stringstyle=\rmfamily\slshape\color[RGB]{128,0,0},
%显示空格
showstringspaces=false,
%长代码自动换行
breaklines=true
}


\bibliographystyle{plain}

\numberwithin{figure}{section}


\begin{document}
    \heiti
    \begin{titlepage}
        \centering
        \rule[1.10cm]{\linewidth}{0cm}

        \includegraphics[width=0.4\linewidth]{figures/Banner}

        \vspace{0.5cm}
        \textbf{\zihao{0} 实验报告}

        \vspace{1cm}
        \textbf{{\Huge 图型结构及其应用}}

        \begin{LARGE}
            \vspace{2cm}
            班\rule{38pt}{0pt}级 \dlmu[6cm]{1603002}\\ \vspace{4pt}
            学\rule{38pt}{0pt}号 \dlmu[6cm]{1160300202}\\ \vspace{4pt}
            姓\rule{38pt}{0pt}名 \dlmu[6cm]{冯云龙}\\ \vspace{4pt}
            指导教师 \dlmu[6cm]{张岩}\\ \vspace{4pt}
            实验日期 \dlmu[6cm]{2017/10/31}\\ \vspace{4pt}
            设计成绩 \dlmu[6cm]{}\\ \vspace{4pt}
            报告成绩 \dlmu[6cm]{}\\ \vspace{4pt}
        \end{LARGE}

        \vfill
        {\huge \textbf{计算机科学与技术学院}}
        % 底部插入当日日期
    \end{titlepage}

    \tableofcontents
    \newpage
    \section{实验目的}
树型结构的建立与遍历。

\section{实验要求及实验环境}
\subsection{实验要求}
\begin{enumerate}
    \item 设计 AVL 的左右链存储结构;
    \item 实现 AVL 左右链存储结构上的插入(建立)、删除、查找和排序算法。
    \item 测试数据以文件形式保存,能反映插入和删除操作的四种旋转,并输出相应的结果。
\end{enumerate}

\subsection{实验环境}
\subsubsection{硬件环境}
\begin{itemize}
    \item CPU: Intel Core i7-6700HQ @ 8x 3.5GHz
    \item RAM: 3516MiB / 7899MiB
\end{itemize}

\subsubsection{软件环境}
\begin{itemize}
    \item Manjaro Linux
\end{itemize}

\subsubsection{开发工具}
\begin{itemize}
    \item GCC
    \item Clion
\end{itemize} %实验目的 实验要求及实验环境
    \section{分配器的设计与实现}
\begin{center}
    总分50分
\end{center}

\subsection{总体设计(10分)}
介绍堆、堆中内存块的组织结构,采用的空闲块、分配块链表/树结构和相应算法等内容。

\begin{figure}[H]
    \begin{minipage}[l]{0.6\linewidth}
        \heiti
        \begin{tikzpicture}
        \draw (0,0) rectangle +(8,8);
        \draw (7,7) -- (7,8);
        \draw (7,0) -- (7,1);
        \draw (0,1) -- (8,1);
        \draw (0,7) -- (8,7);
        \draw (0,3) -- (8,3);
        \draw (4,5) node{有效载荷};
        \draw (4,2) node{填充(可选)};
        \draw (3.5,0.5) node{块大小};
        \draw (3.5,7.5) node{块大小};
        \draw (7.5,0.5) node{a/f};
        \draw (7.5,7.5) node{a/f};
        \end{tikzpicture}
    \end{minipage}
    \begin{minipage}[c]{0.4\linewidth}
        \heiti
        空闲块通过头部的大小字段隐含的连接着,分配器通过遍历堆中所有的块,从而间接的遍历整个空闲块的集合。
        如果找不到合适的内存块,就对堆的大小进行扩展,需要注意的就是对空闲块的合并,使用了边界标记可以是使得这个操作变得很容易。
    \end{minipage}
\end{figure}

\subsection{关键函数设计(40分)}
\subsubsection{int mm\_init(void)函数(5分)}
\textbf{函数功能:}初始化分配器,成功就返回0,否则返回-1。

\textbf{处理流程:}申请一个四字的块,将第一个字节填充置零,第二、三个字分别作为先导块头部与脚部,设为已分配状态,防止之后的内存块错误合并,第四个字节作为新申请的头部,状态为已占用。将heap\_listp指向初始头部与脚部之间载荷(大小为0)位置。

\textbf{要点分析:}
\subsubsection{void mm\_free(void *ptr)函数(5分)}
\textbf{函数功能:}释放参数"ptr"指向的已分配内存块,没有返回值。指针值ptr应该是之前调用mm\_malloc或mm\_realloc返回的值,并且没有释放过。

\textbf{参   数:}需要被释放的指针。

\textbf{处理流程:}获取该块的大小,将头部与脚部的有效位置零表示空闲,合并空闲块。

\textbf{要点分析:}内存块需要重新标记为未使用。

\subsubsection{void *mm\_realloc(void *ptr, size\_t size)(5分)}
\textbf{函数功能:}调用mm\_realloc是为了将ptr所指向内存块(旧块)的大小变为size,并返回新内存块的地址。

\textbf{参   数:}需要重新分批大小的内存块,重新分配的大小。

\textbf{处理流程:}获取需要重新分配的字节大小,与当前块大小比较,如果小于现有块大小,则直接返回,若现有的大小比较小,则查看能否与前后的空闲块进行合并,否则申请新块并复制旧块内容,释放旧块,返回指向新块的指针。

\textbf{要点分析:}需要注意下面几个方面的要求。
\begin{itemize}
    \item 如ptr是空指针NULL,等价于mm\_malloc(size)
    \item 如果参数size为0,等价于mm\_free(ptr)
    \item 如ptr非空, 它应该是之前调用mm\_malloc或mm\_realloc返回的数值,指向一个已分配的内存块。
\end{itemize}
\subsubsection{int mm\_check(void)函数函数(5分)}
\textbf{函数功能:}函数,检查重要的不变量和一致性条件。当且仅当堆是一致的,才能返回非0值。

\textbf{处理流程:}检查空闲列表中的每个块是否都空闲,是否有连续的空闲块没有被合并,是否每个空闲块都在空闲链表中,空闲链表中的指针是否均指向有效的空闲块,分配的块是否有重叠,堆块中的指针是否指向有效的堆地址。若不一致,返回0,若一致,返回1。

\textbf{要点分析:}重点检查如下几个方面。
\begin{enumerate}
    \item 空闲列表中的每个块是否都标识为free(空闲)?
    \item 是否有连续的空闲块没有被合并?
    \item 是否每个空闲块都在空闲链表中?
    \item 空闲链表中的指针是否均指向有效的空闲块?
    \item 分配的块是否有重叠?
    \item 堆块中的指针是否指向有效的堆地址?
\end{enumerate}

\subsubsection{int void *mm\_malloc(size\_t size)函数(10分)}
\textbf{函数功能:}申请有效载荷至少是参数"size"指定大小的内存块,返回该内存块地址首地址(可以使用的区域首地址)。

\textbf{参   数:}申请分配内存块的大小。

\textbf{处理流程:}若传入参数不大于零,返回NULL。否则令新块大小为头部之前填充与对齐后字节大小之和,查找空闲链表中的空闲块,将新块放入,如没有合适空闲块,申请新的内存。

\textbf{要点分析:}申请的整个块应该在对的区间内,并且不能与其他已经分配的块重叠。返回的地址应该是8字节对齐的(地址\%8==0)。
\subsubsection{static void *coalesce(void *bp)函数(10分)}
\textbf{函数功能:}将要回收的空闲块和临近的空闲块(如果有的话)合并成一个大的空闲块。返回合并后的空闲块指针。

\textbf{参   数:}bp是要回收的空闲块指针

\textbf{处理流程:}获取头部和脚部的标识位,以及块的大小,分四种情况。
\begin{enumerate}
    \item 前后都已分配,返回被释放块的指针
    \item 下一块空闲,将块大小扩展为两块之和,释放块的头部和后一块的脚部标识位设置为0
    \item 前一块空闲,将块大小扩展为两块之和,前一块的头部和释放块的脚部标识位设置为0,指针指向前一块
    \item 前后均空闲,将块大小扩展为三块之和,前一块的头部和后一块的脚部标识位设置为0,指针指向前一块
\end{enumerate}

\textbf{要点分析:}每次对块进行扩展,需注意扩展之后对标识位的设置改变,扩展完毕后,需移动指针位置。 %设计思想
    \section{总结}

\subsection{请总结本次实验的收获}
本次实验,对C语言和汇编之间的关系有了更进一步的了解,对于GDB的使用也变得更加熟悉,感觉收获很大。

\subsection{请给出对本次实验内容的建议}
希望能够提前发PPT和实验报告。 %测试结果
    \section{经验体会与不足}
进一步加强了我的编程能力,加深了我对图形存储结构的理解,这一次的实验使用了ncursor库,效果非常好。 %经验体会与不足
    \appendix

\section{源代码}
\lstinputlisting[language = c]{../list.c}
\lstinputlisting[language = c]{../polynomial.c}
\lstinputlisting[language = c]{../polymain.c}
    %附录:源代码

\end{document}