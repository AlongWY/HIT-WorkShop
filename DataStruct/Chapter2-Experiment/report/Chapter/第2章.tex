\section{实验要求及实验环境}
\subsection{实验要求}
\begin{enumerate}
    \item 能够输入多项式(可以按各项的任意输入顺序,建立按指数降幂排列的多项式)和输出多项式(按指数降幂排列),以文件形式输入和输出,并显示。
    \item 能够计算多项式在某一点x=x0的值,其中x0是一个浮点型常量,返回结果为浮点数。
    \item 能够给出计算两个多项式加法、减法、乘法和除法运算的结果多项式,除法运算的结果包括商多项式和余数多项式。
    \item 要求尽量减少乘法和除法运算中间结果的空间占用和结点频繁的分配与回收操作(提示:利用循环链表结构和可用空间表的思想,把循环链表表示的多项式返还给可用空间表,从而解决上述问题)。
\end{enumerate}


\subsection{实验环境}
\subsubsection{硬件环境}
\begin{itemize}
    \item CPU: Intel Core i7-6700HQ @ 8x 3.5GHz
    \item RAM: 3516MiB / 7899MiB
\end{itemize}

\subsubsection{软件环境}
\begin{itemize}
    \item Manjaro Linux
\end{itemize}

\subsubsection{开发工具}
\begin{itemize}
    \item GCC
    \item Clion
\end{itemize}